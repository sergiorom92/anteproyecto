\chapter{Marco teórico}

\section{Conceptos generales de programación}

\subsection{Programación orientada a objetos}

Paradigma programación orientada a objetos.
Representación de la realidad orientada a objetos. (Clases, atributos, métodos).
Conceptos (Herencia, encapsulación, polimorfismo).

\subsection{Modelo Entidad-Relación}

Entidad. La entidad es cualquier clase de objeto o conjunto de elementos presentes o no, en un contexto determinado dado por el sistema de información o las funciones y procesos que se definen en un plan de automatización.
Relación. Vínculo que permite definir una dependencia entre los conjuntos de dos o más entidades. Esto es la relación entre la información contenida en los registros de varias tablas.

Según el estándar las clases de entidad se muestran con rectángulos, las relaciones con diamantes  y la cardinalidad máxima de relación se indica dentro del diamante.

\subsection{Modelo Vista Controlador}

Modelo: Se trata del nucleo funcional que gestiona datos manipulados en aplicación.
Vista: se trata de los componentes destinados a representar la informacion al usuario.
Controlador: este recibe los eventos que provienen del usuario y los traduce en consultas para el modelo y la vista. diamante

\subsection{Diagrama de Clases}

Diagrama utilizado en POO para representar objetos. Representa sus atributos, métodos y relaciones entre objetos.

\subsection{Diagrama de casos de uso}

Permite describir las acciones que un usuario puede desempeñar dentro de un sistema.

\section{Front-End}

Básicamente el software con el que el usuario interactúa. Captura y muestra información. Se piensa realizar el desarrollo de una aplicación móvil híbrida, para garantizar que esta pueda ser ejecutada bajo cualquier sistema operativo móvil (Ej. Android, iOS, etc).

\subsection{AngularJs 2.0}

Framework para desarrollo de aplicaciones web. Propietario de Google.

\subsection{Sass}

Hojas de estilo. Básicamente una versión mejorada de Css. Compilado. Genera CSS, permite heredar reglas CSS.

\subsection{TypeScript}

Mejora de JavaScript. Desarrollado sobre JavaScript. Básicamente aplica paradigma de POO en lenguaje JavaScript. Es compilado, genera código JavaScript.

\subsection{Apache Cordova (Phonegap)}

Framework y conjunto de librerías para desarrollo de aplicaciones híbridas. Haciendo uso de tecnología Web. Librerías que permiten acceso a sensores del dispositivo.

\subsection{Ionic 2.0}

Framework para desarrollo aplicaciones móviles híbridas. Desarrollado sobre el Framework Angular 2.0. 

\section{Back-End}

Básicamente el software con el que el usuario no interactúa directamente pero que es vital para proveer servicios de datos que serán consumidos por el Front-End para ofrecer una funcionalidad.

\subsection{Motor de base de datos PostgreSQL}

PostgreSQL es un motor de bases de datos relacionales (RDBMS) que verifica integridad referencial con gran funcionalidad como base de datos, aunque un poco más lenta que otros motores. Su licencia es tipo BSD. En esta sección describimos brevemente la instalación y uso en un sistema adJ.

\subsection{Java}

Java es un lenguaje de programación de propósito general, concurrente, orientado a objetos que fue diseñado específicamente para tener tan pocas dependencias de implementación como fuera posible. Su intención es permitir que los desarrolladores de aplicaciones escriban el programa una vez y lo ejecuten en cualquier dispositivo (conocido en inglés como WORA, o "write once, run anywhere"), lo que quiere decir que el código que es ejecutado en una plataforma no tiene que ser recompilado para correr en otra. Java es, a partir de 2012, uno de los lenguajes de programación más populares en uso, particularmente para aplicaciones de cliente-servidor de web, con unos 10 millones de usuarios reportados.

\subsection{JPA}

Java Persistence API es un conjunto de clases y métodos que persistentemente almacenar la gran cantidad de datos a una base de datos que es proporcionada por Oracle Corporation.

\subsection{Referencias}


Libro
LANAGAN, D. Y MATSUMOTO, Y.
The Ruby programming language
En el texto: (Flanagan and Matsumoto)
Bibliografía: Flanagan, David and Yukihiro Matsumoto. The Ruby Programming Language. 1st ed. Beijing [etc.]: O'Reilly, 2008. Print.


Libro
SEIDL, M.
UML @ classroom
En el texto: (Seidl)
Bibliografía: Seidl, Martina. UML @ Classroom. 1st ed. Cham: Springer, 2015. Print.


Libro
SIERRA, K. Y BATES, B.
Head first Java
En el texto: (Sierra and Bates)
Bibliografía: Sierra, Kathy and Bert Bates. Head First Java. 1st ed. Beijing [u.a.]: O'Reilly, 2008. Print.


Libro
THAKUR, V.
ASP. NET 3. 5 Application Architecture and Design
En el texto: (Thakur)
Bibliografía: Thakur, Vivek. ASP. NET 3. 5 Application Architecture And Design. 1st ed. Birmingham: Packt Publishing, Limited, 2008. Print.